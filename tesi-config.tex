%**************************************************************
% file contenente le impostazioni della tesi
%**************************************************************

%**************************************************************
% Frontespizio
%**************************************************************

% Autore
\newcommand{\myName}{Federico Silvio Busetto}       
\newcommand{\matricola}{1026925}               

% Azienda
\newcommand{\myCompany}{IKS s.r.l.}
% Titolo della tesi              
\newcommand{\myTitle}{Un sistema di continuous building basato su Jenkins e tecnologie di containerizzazione}

% Tipo di tesi                   
\newcommand{\myDegree}{Tesi di laurea triennale}

% Università             
\newcommand{\myUni}{Università degli Studi di Padova}

% Facoltà       
\newcommand{\myFaculty}{Corso di Laurea in Informatica}

% Dipartimento
\newcommand{\myDepartment}{Dipartimento di Matematica "Tullio Levi-Civita"}

% Titolo del relatore
\newcommand{\profTitle}{Prof.}

% Relatore  e tutor aziendale
\newcommand{\myProf}{Gilberto Filè}
\newcommand{\myTutor}{Massimo Celegato}

% Luogo
\newcommand{\myLocation}{Padova}

% Anno accademico
\newcommand{\myAA}{2016-2017}

% Data discussione
\newcommand{\myTime}{07 Dicembre 2017}


%**************************************************************
% Impostazioni di impaginazione
% see: http://wwwcdf.pd.infn.it/AppuntiLinux/a2547.htm
%**************************************************************

\pagestyle{fancy}
\fancyhf{}
\setlength{\headheight}{1cm} 
\setlength{\parindent}{0pt}   % larghezza rientro della prima riga
\setlength{\parskip}{0pt}     % distanza tra i paragrafi
\renewcommand{\chaptermark}[1]{\markboth{#1}{}} 


\fancypagestyle{IHA-fancy-style}{%
    \fancyhf{}% Clear header and footer
    \fancyhead[LE,RO]{\slshape \rightmark}
    \fancyhead[LO,RE]{\slshape \leftmark}
    \fancyfoot[LE,RO]{\thepage} % Left side on Even pages; Right side on Odd pages
    \renewcommand{\headrulewidth}{0.4pt}% Line at the header visible
    \renewcommand{\footrulewidth}{0.4pt}% Line at the footer visible
}

% Redefine the plain page style
\fancypagestyle{plain}{%
   \fancyhf{}%
   \fancyfoot[LE,RO]{\thepage} % Left side on Even pages; Right side on Odd pages
   \renewcommand{\headrulewidth}{0pt}% Line at the header invisible
   \renewcommand{\footrulewidth}{0.4pt}% Line at the footer visible
}

%***INSERIMENTO DI NUOVE SOTTOSEZIONI
\setcounter{secnumdepth}{5}		% mostra nel documento fino al livello 8 (1.2.3.4)
\setcounter{tocdepth}{5}			% mostra nell'indice fino al livello 8 (1.2.3.4)

%**************************************************************
% Impostazioni di biblatex
%**************************************************************
\bibliography{bibliografia} % database di biblatex 
%\addbibresource{bibliografia.bib}

\defbibheading{bibliography} {
    \cleardoublepage
    \phantomsection 
    \addcontentsline{toc}{chapter}{\bibname}
    \chapter*{\bibname\markboth{\bibname}{\bibname}}
}

\setlength\bibitemsep{1.5\itemsep} % spazio tra entry

\DeclareBibliographyCategory{opere}
\DeclareBibliographyCategory{web}

\addtocategory{opere}{womak:lean-thinking}
\addtocategory{web}{site:agile-manifesto}

\defbibheading{opere}{\section*{Riferimenti bibliografici}}
\defbibheading{web}{\section*{Siti Web consultati}}


%**************************************************************
% Impostazioni di caption
%**************************************************************
\captionsetup{
    tableposition=top,
    figureposition=bottom,
    font=small,
    format=hang,
    labelfont=bf
}

%**************************************************************
% Impostazioni di glossaries
%**************************************************************

%**************************************************************
% Acronimi
%**************************************************************
\renewcommand{\acronymname}{Acronimi e abbreviazioni}

\newacronym[description={\glslink{apig}{Application Program Interface}}]
    {api}{API}{Application Program Interface}

\newacronym[description={\glslink{bash}{Burn Again Shell}}]
    {bash}{Bash}{Burn Again Shell}

\newacronym[description={\glslink{CentOS}{Community enterprise Operating System}}]
    {CentOS}{CentOS}{Community enterprise Operating System}
    
\newacronym[description={\glslink{CIg}{Configuration Item}}]
    {CI}{CI}{Configuration Item}

\newacronym[description={\glslink{gnug}{GNU is Not Unix}}]
    {GNU}{GNU}{}

\newacronym[description={\glslink{ide}{Integrated Development Environment}}]
{ide}{IDE}{Integrated Development Environment}

\newacronym[description={\glslink{iot}{Internet of Things}}]
{iot}{IoT}{Internet of Things}

\newacronym[description={\glslink{jdk}{Java Development Kit}}]
    {jdk}{JDK}{Java Development Kit}
    
\newacronym[description={\glslink{json}{Javascript Object Notation}}]
    {json}{JSON}{Javascript Object Notation}
    
\newacronym[description={\glslink{jvm}{Java Virtual Machine}}]
{jvm}{JVM}{Java Virtual Machine}

\newacronym[description={\glslink{pom}{Project Object Model}}]
    {pom}{POM}{Project Object Model}

\newacronym[description={\glslink{umlg}{Unified Modeling Language}}]
    {uml}{UML}{Unified Modeling Language}
    
\newacronym[description={\glslink{xml}{eXtensible Markup Language}}]
    {xml}{XML}{eXtensible Markup Language}

%**************************************************************
% Glossario
%**************************************************************
%\renewcommand{\glossaryname}{Glossario}

\newglossaryentry{apig}
{
    name=\glslink{api}{API},
    text=Application Program Interface,
    sort=api,
    description={in informatica con il termine \emph{Application Programming Interface API} (ing. interfaccia di programmazione di un'applicazione) si indica ogni insieme di procedure disponibili al programmatore, di solito raggruppate a formare un set di strumenti specifici per l'espletamento di un determinato compito all'interno di un certo programma. La finalità è ottenere un'astrazione, di solito tra l'hardware e il programmatore o tra software a basso e quello ad alto livello semplificando così il lavoro di programmazione}
}

\newglossaryentry{umlg}
{
    name=\glslink{uml}{UML},
    text=UML,
    sort=uml,
    description={In ingegneria del software \emph{UML, Unified Modeling Language} (ing. linguaggio di modellazione unificato) è un linguaggio di modellazione e specifica basato sul paradigma object-oriented. L'\emph{UML} svolge un'importantissima funzione di ``lingua franca'' nella comunità della progettazione e programmazione a oggetti. Gran parte della letteratura di settore usa tale linguaggio per descrivere soluzioni analitiche e progettuali in modo sintetico e comprensibile a un vasto pubblico}
}

\newglossaryentry{agent}
{
    name=\glslink{agent}{agent},
    text=agent,
    sort=agent,
    description={Sinonimo di slave. Nodo fisico, virtuale o container, che esegue compiti definiti e diretti da un'istanza \textit{master}}
}

\newglossaryentry{agile}
{
    name=\glslink{agile}{agile},
    text=agile,
    sort=agile,
    description={Insieme di metodi di sviluppo del software emersi a partire dai primi anni 2000, fondati su un insieme di principi comuni, direttamente o indirettamente derivati dai princìpi del \textit{Manifesto per lo sviluppo agile del software  \footnote{\url{http://agilemanifesto.org/}}}}
}

\newglossaryentry{application-server}
{
    name=\glslink{application-server}{application-server},
    text=Application Server,
    sort=application server,
    description={Tipologia di server che fornisce l'infrastruttura e le funzionalità di supporto, sviluppo ed esecuzione di applicazioni nonché altri componenti server in un contesto distribuito}
}

\newglossaryentry{artefatto}
{
    name=\glslink{artefatto}{artefatto},
    text=artefatto,
    sort=artefatto,
    description={Un prodotto immutabile, generato durante una \textit{build} che verrà opportunamente archiviato in un \textit{repository} e reso disponibile a determinati utenti per usi specifici (es. rilascio, testing)}
}

\newglossaryentry{baseline}
{
    name=\glslink{baseline}{baseline},
    text=baseline,
    sort=baseline,
    description={Nel ciclo di vita di un progetto, punto d'arrivo tecnico dal quale non si retrocede, sul quale calcolare l'avanzamento del lavoro in un progetto}
}

\newglossaryentry{blockchain}
{
    name=\glslink{blockchain}{blockchain},
    text=blockchain,
    sort=blockchain,
    description={Base di dati distribuita, introdotta dalla valuta Bitcoin che mantiene in modo continuo una lista crescente di record, i quali fanno riferimento a record precedenti presenti nella lista stessa in modo da resistere a manomissioni}
}

\newglossaryentry{build}
{
    name=\glslink{build}{build},
    text=build,
    sort=build,
    description={La costruzione, osservabile e tangibile di una componente software a partire dal suo codice sorgente.\\ 
    Risultato della singola esecuzione di un \textit{project}, in \textit{Jenkins}}
}

\newglossaryentry{bytecode}
{
    name=\glslink{bytecode}{bytecode},
    text=bytecode,
    sort=bytecode,
    description={Linguaggio intermedio tra il codice sorgente e il linguaggio macchina}
}

\newglossaryentry{CIg}
{
    name=\glslink{CI}{CI},
    text=Configuration Item,
    sort=configuration item,
    description={Singolo elemento sottoposto ad attività di gestione di configurazione. Ogni CI è identificato univocamente da:
        \begin{itemize}
            \item Codice Identificativo (ID);
            \item Nome;
            \item Data;
            \item Autore;
            \item Registro delle modifiche;
            \item Stato corrente;
        \end{itemize}}
}

\newglossaryentry{cloud}
{
    name=\glslink{cloud}{cloud},
    text=cloud,
    sort=cloud,
    description={Abbreviazione di cloud computing. Paradigma informatico per l'utilizzo di risorse hardware o software in remoto, tramite la disponibilità on demand di risorse preesistenti e configurabili.}
}

\newglossaryentry{container}
{
    name=\glslink{container}{container},
    text=container,
    sort=container,
    description={Processo isolato dal resto del sistema che rappresenta l'istanza in esecuzione di una data immagine, secondo opportune configurazioni aggiuntive}
}

\newglossaryentry{continuous building}
{
    name=\glslink{continuous building}{continuous building},
    text=continuous building,
    sort=continuous building,
    description={In ingegneria del software per \emph{continuous building}, (spesso identificata come ``Build automation'') si intende il processo di automazione dell'intero ciclo di build a partire dalla compilazione del codice sorgente, al garantire che i risultati siano sempre ripetibili e conformi alle attese, all'esecuzione dei test e al rendere disponibile gli artefatti risultanti. Tale processo è una parte consistente dell'\textit{integrazione continua}}
}

\newglossaryentry{debugger}
{
    name=\glslink{debugger}{debugger},
    text=debugger,
    sort=debugger,
    description={Strumento per agevolare l'individuazione e permettere un'agevole rimozione di errori software, comunemente chiamati bug}
}

\newglossaryentry{deploy}
{
    name=\glslink{deploy}{deploy},
    text=deploy,
    sort=deploy,
    description={L'insieme delle attività che permettono ad una componente software di essere disponibile ad un determinato uso}
}

\newglossaryentry{gnug}
{
    name=\glslink{GNU}{GNU},
    text=GNU,
    sort=gnu,
    description={Sistema operativo Unix-like, ideato nel 1984 da Richard Stallman e promosso dalla Free Software Foundation, allo scopo di ottenere un sistema operativo completo utilizzando esclusivamente software libero}
}

\newglossaryentry{integrazione-continua}
{
    name=\glslink{integrazione-continua}{integrazione-continua},
    text=integrazione continua,
    sort=integrazione continua,
    description={Insieme delle attività che permettono un allineamento frequente degli ambienti di lavoro di più sviluppatori}
}

\newglossaryentry{immagine}
{
    name=\glslink{immagine}{immagine},
    text=immagine,
    sort=immagine,
    description={Base fondante e immutabile utilizzata per la costruzione di \textit{container}}
}
 
\newglossaryentry{Java EE}
{
    name=\glslink{Java EE}{Java EE},
    text=Java Enterprise Edition,
    sort=Java EE,
    description={Insieme di specifiche le cui implementazioni vengono sviluppate in linguaggio di programmazione Java con ampio utilizzo nella programmazione Web}
}

\newglossaryentry{Linux}
{
    name=\glslink{Linux}{Linux},
    text=Linux,
    sort=linux,
    description={Famiglia di sistemi operativi di tipo Unix-like, pubblicati sotto varie possibili distribuzioni, aventi la caratteristica comune di utilizzare come nucleo il kernel Linux}
}

\newglossaryentry{nodo}
{
    name=\glslink{nodo}{nodo},
    text=nodo,
    sort=nodo,
    description={Insieme di specifiche le cui implementazioni vengono sviluppate in linguaggio di programmazione Java con ampio utilizzo nella programmazione Web}
}

\newglossaryentry{machine-learning}
{
    name=\glslink{machine-learning}{machine-learning},
    text=machine-learning,
    sort=machine-learning,
    description={Campo dell'informatica che fornisce alle macchine l'abilità di imparare, nello svolgere determinati compiti, senza essere programmate esplicitamente}
}

\newglossaryentry{master}
{
    name=\glslink{master}{master},
    text=master,
    sort=master,
    description={L'unità centrale di coordinazione dei processi che possiede la configurazione, carica i plugin e gestisce tutti gli aspetti di esecuzione dei vari \textit{projects}}
}

\newglossaryentry{middleware}
{
    name=\glslink{middleware}{middleware},
    text=middleware,
    sort=middleware,
    description={Insieme di programmi informatici che fungono da intermediari tra diverse applicazioni e componenti software}
}

\newglossaryentry{plugin}
{
    name=\glslink{plugin}{plugin},
    text=plugin,
    sort=plugin,
    description={Componente software che fornisce funzionalità aggiuntive ad estensione di un determinato prodotto software preesistente ma separatamente da esso}
}

\newglossaryentry{project}
{
    name=\glslink{project}{project},
    text=project,
    sort=project,
    description={Una configurazione definita da un utente, che descrive un lavoro che Jenkins dovrà svolgere, come ad esempio costruire una componente software, eseguire uno script, ecc... A volte riferito come \textit{job}}
}

\newglossaryentry{Red Hat}
{
    name=\glslink{Red Hat}{Red Hat},
    text=Red Hat,
    sort=red hat,
    description={Società multinazionale statunitense che si dedica allo sviluppo software e al supporto di software libero e open source in ambiente aziendale}
}

\newglossaryentry{repository}
{
    name=\glslink{repository}{repository},
    text=repository,
    sort=repository,
    description={Base di dati centralizzata nella quale risiedono, individualmente, tutti i \gls{CI} di ogni \gls{baseline} nella loro storia completa}
}

\newglossaryentry{slave}
{
    name=\glslink{slave}{slave},
    text=slave,
    sort=slave,
    description={Letteralmente, schiavo. In Jenkins rappresenta un nodo sul quale vengono eseguiti più compiti, definiti sotto forma di \textit{project}, secondo la direzione di un \textit{master}. A volte definito come \textit{Agent}}
}

\newglossaryentry{system integration}
{
    name=\glslink{system integration}{system integration},
    text=system integration,
    sort=system integration,
    description={Processo che consiste nel collegare assieme differenti sistemi e applicazioni software, a livello fisico o funzionale in modo che agiscano come un insieme coordinato e coeso}
} % database di termini
\makeglossaries


%**************************************************************
% Impostazioni di graphicx
%**************************************************************
\graphicspath{{immagini/}} % cartella dove sono riposte le immagini


%**************************************************************
% Impostazioni di hyperref
%**************************************************************
\hypersetup{
    %hyperfootnotes=false,
    %pdfpagelabels,
    %draft,	% = elimina tutti i link (utile per stampe in bianco e nero)
    colorlinks=true,
    linktocpage=true,
    pdfstartpage=1,
    pdfstartview=FitV,
    % decommenta la riga seguente per avere link in nero (per esempio per la stampa in bianco e nero)
    %colorlinks=false, linktocpage=false, pdfborder={0 0 0}, pdfstartpage=1, pdfstartview=FitV,
    breaklinks=true,
    pdfpagemode=UseNone,
    pageanchor=true,
    pdfpagemode=UseOutlines,
    plainpages=false,
    bookmarksnumbered,
    bookmarksopen=true,
    bookmarksopenlevel=1,
    hypertexnames=true,
    pdfhighlight=/O,
    %nesting=true,
    %frenchlinks,
    urlcolor=webbrown,
    linkcolor=RoyalBlue,
    citecolor=webgreen,
    %pagecolor=RoyalBlue,
    %urlcolor=Black, linkcolor=Black, citecolor=Black, %pagecolor=Black,
    pdftitle={\myTitle},
    pdfauthor={\textcopyright\ \myName, \myUni, \myFaculty},
    pdfsubject={},
    pdfkeywords={},
    pdfcreator={pdfLaTeX},
    pdfproducer={LaTeX}
}

%**************************************************************
% Impostazioni di itemize
%**************************************************************
\renewcommand{\labelitemi}{$\ast$}

%\renewcommand{\labelitemi}{$\bullet$}
%\renewcommand{\labelitemii}{$\cdot$}
%\renewcommand{\labelitemiii}{$\diamond$}
%\renewcommand{\labelitemiv}{$\ast$}


%**************************************************************
% Impostazioni di listings
%**************************************************************

%Dockerfile
\lstdefinelanguage{docker}{
    keywords={FROM, RUN, COPY, ADD, ENTRYPOINT, CMD,  ENV, WORKDIR, EXPOSE, LABEL, USER, VOLUME, STOPSIGNAL, ONBUILD, MAINTAINER},
    keywordstyle=\color{blue}\bfseries,
    identifierstyle=\color{black},
    sensitive=false,
    comment=[l]{\#},
    numbers=left,
    stepnumber=1,
    commentstyle=\color{mygreen}\ttfamily,
    stringstyle=\color{red}\ttfamily,
    morestring=[b]',
    morestring=[b]"
}

\lstdefinelanguage{docker-shell}{
    keywords={docker, cp, sudo, rm, tar, run, build, volume, network, inspect},
    keywordstyle=\color{blue}\bfseries,
    identifierstyle=\color{black},
    sensitive=false,
    comment=[l]{\#},
    numbers=left,
    stepnumber=1,
    commentstyle=\color{mygreen}\ttfamily,
    stringstyle=\color{red}\ttfamily,
    morestring=[b]',
    morestring=[b]"
}

%C++
\lstset{
    language=[LaTeX]Tex,
    keywordstyle=\color{RoyalBlue}, %\bfseries,
    basicstyle=\small\ttfamily,
    %identifierstyle=\color{NavyBlue},
    commentstyle=\color{Green}\ttfamily,
    stringstyle=\rmfamily,
    numbers=none, %left,%
    numberstyle=\scriptsize, %\tiny
    stepnumber=5,
    numbersep=8pt,
    showstringspaces=false,
    breaklines=true,
    frameround=ftff,
    frame=single
} 


%**************************************************************
% Impostazioni di xcolor
%**************************************************************
\definecolor{webgreen}{rgb}{0,.5,0}
\definecolor{webbrown}{rgb}{.6,0,0}
\definecolor{mygreen}{rgb}{0,0.6,0}
\definecolor{darkgreen}{RGB}{0,100,0}
\definecolor{mygray}{rgb}{0.5,0.5,0.5}
\definecolor{mymauve}{rgb}{0.58,0,0.82}
\definecolor{lightGray}{RGB}{150, 150, 150}
\definecolor{darkorange}{RGB}{255, 140, 0}
\definecolor{darkblue}{rgb}{0.07, 0.04, 0.56}
\definecolor{lightblue}{RGB}{176, 196, 222}
\definecolor{tableHeaderRed}{RGB}{211, 47, 0}
\definecolor{tableHeaderBlue}{RGB}{2, 49, 103}
\definecolor{tableLineOne}{RGB}{245, 245, 245}
\definecolor{tableLineTwo}{RGB}{224, 224, 224}
\definecolor{Maroon}{rgb}{0.48, 0.07, 0.07}

%**************************************************************
% Impostazioni quote
%**************************************************************

\newcommand*\openquote{\makebox(25,-22){\scalebox{5}{``}}}
\newcommand*\closequote{\makebox(25,-22){\scalebox{5}{''}}}
\colorlet{shadecolor}{Azure}

\makeatletter
\newif\if@right
\def\shadequote{\@righttrue\shadequote@i}
\def\shadequote@i{\begin{snugshade}\begin{quote}\openquote}
        \def\endshadequote{%
            \if@right\hfill\fi\closequote\end{quote}\end{snugshade}}
\@namedef{shadequote*}{\@rightfalse\shadequote@i}
\@namedef{endshadequote*}{\endshadequote}
\makeatother

%**************************************************************
% Altro
%**************************************************************

\newcommand{\omissis}{[\dots\negthinspace]} % produce [...]

% eccezioni all'algoritmo di sillabazione
\hyphenation
{
    ma-cro-istru-zio-ne
}

\newcommand{\sectionname}{sezione}
\addto\captionsitalian{\renewcommand{\figurename}{Figura}
                       \renewcommand{\tablename}{Tabella}}

\newcommand{\glsfirstoccur}{\ap{{[g]}}}

\newcommand{\intro}[1]{\emph{\textsf{#1}}}

%**************************************************************
% Environment per ``rischi''
%**************************************************************
\newcounter{riskcounter}                % define a counter
\setcounter{riskcounter}{0}             % set the counter to some initial value

%%%% Parameters
% #1: Title
\newenvironment{risk}[1]{
    \refstepcounter{riskcounter}        % increment counter
    \par \noindent                      % start new paragraph
    \textbf{\arabic{riskcounter}. #1}   % display the title before the 
                                        % content of the environment is displayed 
}{
    \par\medskip
}

\newcommand{\riskname}{Rischio}

\newcommand{\riskdescription}[1]{\textbf{\\Descrizione:} #1.}

\newcommand{\risksolution}[1]{\textbf{\\Soluzione:} #1.}

%**************************************************************
% Environment per ``use case''
%**************************************************************
\newcounter{usecasecounter}             % define a counter
\setcounter{usecasecounter}{0}          % set the counter to some initial value

%%%% Parameters
% #1: ID
% #2: Nome
\newenvironment{usecase}[2]{
    \renewcommand{\theusecasecounter}{\usecasename #1}  % this is where the display of 
                                                        % the counter is overwritten/modified
    \refstepcounter{usecasecounter}             % increment counter
    \vspace{10pt}
    \par \noindent                              % start new paragraph
    {\large \textbf{\usecasename #1: #2}}       % display the title before the 
                                                % content of the environment is displayed 
    \medskip
}{
    \medskip
}

\newcommand{\usecasename}{UC}

\newcommand{\usecaseactors}[1]{\textbf{\\Attori Principali:} #1. \vspace{4pt}}
\newcommand{\usecasepre}[1]{\textbf{\\Precondizioni:} #1. \vspace{4pt}}
\newcommand{\usecasedesc}[1]{\textbf{\\Descrizione:} #1. \vspace{4pt}}
\newcommand{\usecasepost}[1]{\textbf{\\Postcondizioni:} #1. \vspace{4pt}}
\newcommand{\usecasealt}[1]{\textbf{\\Scenario Alternativo:} #1. \vspace{4pt}}

%**************************************************************
% Environment per ``namespace description''
%**************************************************************

\newenvironment{namespacedesc}{
    \vspace{10pt}
    \par \noindent                              % start new paragraph
    \begin{description} 
}{
    \end{description}
    \medskip
}

\newcommand{\classdesc}[2]{\item[\textbf{#1:}] #2}