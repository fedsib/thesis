% !TEX encoding = UTF-8
% !TEX TS-program = pdflatex
% !TEX root = ../tesi.tex

%**************************************************************
% Sommario
%**************************************************************
\cleardoublepage
\phantomsection
\pdfbookmark{Sommario}{Sommario}
\begingroup
\let\clearpage\relax
\let\cleardoublepage\relax
\let\cleardoublepage\relax

\chapter*{Sommario}

Il presente documento descrive il lavoro svolto durante il periodo di stage, della durata di circa trecento ore, dal laureando \myName\ presso l'azienda \myCompany.\\

Lo scopo principale dello stage è stato quello di realizzare un sistema di continuous building, in grado di prendere in ingresso il codice sorgente di un'applicazione d'esempio, effettuarne la compilazione, e produrre in uscita un artefatto pronto per il deploy.

Era inoltre richiesto che tale sistema prevedesse l'utilizzo di tecnologie di containerizzazione, al fine di valutare gli effettivi vantaggi della loro adozione.

Particolare attenzione è stata posta agli aspetti legati all'esercibilità, quali: monitoraggio, aggiornamento, backup e ripristino della piattaforma.\\

Gli obiettivi da raggiungere erano molteplici:\\
In primo luogo era richiesta la progettazione e la definizione dell'architettura sia di alto livello che di dettaglio di tale sistema, e la sua implementazione.

In secondo luogo è stato richiesto uno studio di fattibilità sulle modalità di comunicazione tra nodi, approfondendo in particolare le comunicazioni tra nodi container in un ambiente distribuito.

Infine era richiesto lo sviluppo di un plugin base per Jenkins, da utilizzare come modello di riferimento per l'eventuale implementazione di nuove funzionalità a livello di piattaforma.

%\vfill
%
%\selectlanguage{english}
%\pdfbookmark{Abstract}{Abstract}
%\chapter*{Abstract}
%
%\selectlanguage{italian}

\endgroup			

\vfill

