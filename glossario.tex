
%**************************************************************
% Acronimi
%**************************************************************
\renewcommand{\acronymname}{Acronimi e abbreviazioni}

\newacronym[description={\glslink{apig}{Application Program Interface}}]
    {api}{API}{Application Program Interface}

\newacronym[description={\glslink{bash}{The Burn-Again Shell}}]
    {bash}{Bash}{The Burn-Again Shell}

\newacronym[description={\glslink{cbse}{Component Based Software Engineering}}]
{cbse}{CBSE}{Component Based Software Engineering}

\newacronym[description={\glslink{CentOS}{Community enterprise Operating System}}]
    {CentOS}{CentOS}{Community enterprise Operating System}
    
\newacronym[description={\glslink{CIg}{Configuration Item}}]
    {CI}{CI}{Configuration Item}
    
\newacronym[description={\glslink{cvs}{Concurrent Versions System}}]
    {cvs}{CVS}{Concurrent Versions System}

\newacronym[description={\glslink{gnug}{GNU is Not Unix}}]
    {GNU}{GNU}{}

\newacronym[description={\glslink{ide}{Integrated Development Environment}}]
{ide}{IDE}{Integrated Development Environment}

\newacronym[description={\glslink{iot}{Internet of Things}}]
{iot}{IoT}{Internet of Things}

\newacronym[description={\glslink{jdk}{Java Development Kit}}]
    {jdk}{JDK}{Java Development Kit}

\newacronym[description={\glslink{jnlpg}{Java Network Launching Protocol}}]
{jnlp}{JNLP}{Java Network Launching Protocol}
    
\newacronym[description={\glslink{json}{Javascript Object Notation}}]
    {json}{JSON}{Javascript Object Notation}
    
\newacronym[description={\glslink{jvm}{Java Virtual Machine}}]
{jvm}{JVM}{Java Virtual Machine}

\newacronym[description={\glslink{jws}{Java Web Start}}]
{jws}{JWS}{Java Web Start}

\newacronym[description={\glslink{lts}{Long Term Support}}]
{lts}{LTS}{Long Term Support}

\newacronym[description={\glslink{pom}{Project Object Model}}]
    {pom}{POM}{Project Object Model}

\newacronym[description={\glslink{sshg}{Secure Shell}}]
{ssh}{SSH}{Secure Shell}

\newacronym[description={\glslink{umlg}{Unified Modeling Language}}]
    {uml}{UML}{Unified Modeling Language}
    
\newacronym[description={\glslink{xml}{eXtensible Markup Language}}]
    {xml}{XML}{eXtensible Markup Language}

%**************************************************************
% Glossario
%**************************************************************
%\renewcommand{\glossaryname}{Glossario}

\newglossaryentry{apig}
{
    name=\glslink{api}{API},
    text=Application Program Interface,
    sort=api,
    description={in informatica con il termine \emph{Application Programming Interface API} (ing. interfaccia di programmazione di un'applicazione) si indica ogni insieme di procedure disponibili al programmatore, di solito raggruppate a formare un set di strumenti specifici per l'espletamento di un determinato compito all'interno di un certo programma. La finalità è ottenere un'astrazione, di solito tra l'hardware e il programmatore o tra software a basso e quello ad alto livello semplificando così il lavoro di programmazione}
}

\newglossaryentry{umlg}
{
    name=\glslink{uml}{UML},
    text=UML,
    sort=uml,
    description={In ingegneria del software \emph{UML, Unified Modeling Language} (ing. linguaggio di modellazione unificato) è un linguaggio di modellazione e specifica basato sul paradigma object-oriented. L'\emph{UML} svolge un'importantissima funzione di ``lingua franca'' nella comunità della progettazione e programmazione a oggetti. Gran parte della letteratura di settore usa tale linguaggio per descrivere soluzioni analitiche e progettuali in modo sintetico e comprensibile a un vasto pubblico}
}

\newglossaryentry{agent}
{
    name=\glslink{agent}{agent},
    text=agent,
    sort=agent,
    description={Sinonimo di slave. Nodo fisico, virtuale o container, che esegue compiti definiti e diretti da un'istanza \textit{master}}
}

\newglossaryentry{agile}
{
    name=\glslink{agile}{agile},
    text=agile,
    sort=agile,
    description={Insieme di metodi di sviluppo del software emersi a partire dai primi anni 2000, fondati su un insieme di principi comuni, direttamente o indirettamente derivati dai princìpi del \textit{Manifesto per lo sviluppo agile del software  \footnote{\url{http://agilemanifesto.org/}}}}
}

\newglossaryentry{application-server}
{
    name=\glslink{application-server}{application-server},
    text=Application Server,
    sort=application server,
    description={Tipologia di server che fornisce l'infrastruttura e le funzionalità di supporto, sviluppo ed esecuzione di applicazioni nonché altri componenti server in un contesto distribuito}
}

\newglossaryentry{artefatto}
{
    name=\glslink{artefatto}{artefatto},
    text=artefatto,
    sort=artefatto,
    description={Un prodotto immutabile, generato durante una \textit{build} che verrà opportunamente archiviato in un \textit{repository} e reso disponibile a determinati utenti per usi specifici (es. rilascio, testing)}
}

\newglossaryentry{baseline}
{
    name=\glslink{baseline}{baseline},
    text=baseline,
    sort=baseline,
    description={Nel ciclo di vita di un progetto, punto d'arrivo tecnico dal quale non si retrocede, sul quale calcolare l'avanzamento del lavoro in un progetto}
}

\newglossaryentry{blockchain}
{
    name=\glslink{blockchain}{blockchain},
    text=blockchain,
    sort=blockchain,
    description={Base di dati distribuita, introdotta dalla valuta Bitcoin che mantiene in modo continuo una lista crescente di record, i quali fanno riferimento a record precedenti presenti nella lista stessa in modo da resistere a manomissioni}
}

\newglossaryentry{build}
{
    name=\glslink{build}{build},
    text=build,
    sort=build,
    description={La costruzione, osservabile e tangibile di una componente software a partire dal suo codice sorgente.\\ 
    Risultato della singola esecuzione di un \textit{project}, in \textit{Jenkins}}
}

\newglossaryentry{bytecode}
{
    name=\glslink{bytecode}{bytecode},
    text=bytecode,
    sort=bytecode,
    description={Linguaggio intermedio tra il codice sorgente e il linguaggio macchina}
}

\newglossaryentry{CIg}
{
    name=\glslink{CI}{CI},
    text=Configuration Item,
    sort=configuration item,
    description={Singolo elemento sottoposto ad attività di gestione di configurazione. Ogni CI è identificato univocamente da:
        \begin{itemize}
            \item Codice Identificativo (ID);
            \item Nome;
            \item Data;
            \item Autore;
            \item Registro delle modifiche;
            \item Stato corrente;
        \end{itemize}}
}

\newglossaryentry{cloud}
{
    name=\glslink{cloud}{cloud},
    text=cloud,
    sort=cloud,
    description={Abbreviazione di cloud computing. Paradigma informatico per l'utilizzo di risorse hardware o software in remoto, tramite la disponibilità on demand di risorse preesistenti e configurabili.}
}

\newglossaryentry{container}
{
    name=\glslink{container}{container},
    text=container,
    sort=container,
    description={Processo isolato dal resto del sistema che rappresenta l'istanza in esecuzione di una data immagine, secondo opportune configurazioni aggiuntive}
}

\newglossaryentry{continuous building}
{
    name=\glslink{continuous building}{continuous building},
    text=continuous building,
    sort=continuous building,
    description={In ingegneria del software per \emph{continuous building}, (spesso identificata come ``Build automation'') si intende il processo di automazione dell'intero ciclo di build a partire dalla compilazione del codice sorgente, al garantire che i risultati siano sempre ripetibili e conformi alle attese, all'esecuzione dei test e al rendere disponibile gli artefatti risultanti. Tale processo è una parte consistente dell'\textit{integrazione continua}}
}

\newglossaryentry{debugger}
{
    name=\glslink{debugger}{debugger},
    text=debugger,
    sort=debugger,
    description={Strumento per agevolare l'individuazione e permettere un'agevole rimozione di errori software, comunemente chiamati bug}
}

\newglossaryentry{demone}
{
    name=\glslink{demone}{demone},
    text=demone,
    sort=demone,
    description={Programma o servizio in esecuzione in modalità non interattiva con l'utente}
}

\newglossaryentry{executor}
{
    name=\glslink{executor}{executor},
    text=esecutori,
    sort=esecutori,
    description={Slot dedicato all'esecuzione di lavori definiti da una pipeline o da un project su un nodo. Un nodo può avere zero o più esecutori configurati, i quali corrispondono al numero di project o pipeline concorrenti eseguibili su quel nodo.}
}

\newglossaryentry{deploy}
{
    name=\glslink{deploy}{deploy},
    text=deploy,
    sort=deploy,
    description={L'insieme delle attività che permettono ad una componente software di essere disponibile ad un determinato uso}
}

\newglossaryentry{gnug}
{
    name=\glslink{GNU}{GNU},
    text=GNU,
    sort=gnu,
    description={Sistema operativo Unix-like, ideato nel 1984 da Richard Stallman e promosso dalla Free Software Foundation, allo scopo di ottenere un sistema operativo completo utilizzando esclusivamente software libero}
}

\newglossaryentry{host}
{
    name=\glslink{host}{host},
    text=host,
    sort=host,
    description={Nodo connesso alla rete, il quale ospita un'applicazione fornendone le risorse necessarie alla sua esecuzione}
}

\newglossaryentry{integrazione-continua}
{
    name=\glslink{integrazione-continua}{integrazione-continua},
    text=integrazione continua,
    sort=integrazione continua,
    description={Insieme delle attività che permettono un allineamento frequente degli ambienti di lavoro di più sviluppatori}
}

\newglossaryentry{immagine}
{
    name=\glslink{immagine}{immagini},
    text=immagine,
    sort=immagine,
    description={Base fondante e immutabile utilizzata per la costruzione di \textit{container}}
}

\newglossaryentry{jar}
{
    name=\glslink{jar}{jar},
    text=jar,
    sort=jar,
    description={Archivio dati compresso usato per distribuire raccolte di classi Java.}
}
 
\newglossaryentry{Java EE}
{
    name=\glslink{Java EE}{Java EE},
    text=Java Enterprise Edition,
    sort=Java EE,
    description={Insieme di specifiche le cui implementazioni vengono sviluppate in linguaggio di programmazione Java con ampio utilizzo nella programmazione Web}
}

\newglossaryentry{jnlpg}
{
    name=\glslink{jnlp}{JNLP},
    text=JNLP,
    sort=jnlp,
    description={Protocollo, definito da uno schema XML, che specifica la modalità con cui lanciare le applicazioni Java Web Start}
}

\newglossaryentry{Linux}
{
    name=\glslink{Linux}{Linux},
    text=Linux,
    sort=linux,
    description={Famiglia di sistemi operativi di tipo Unix-like, pubblicati sotto varie possibili distribuzioni, aventi la caratteristica comune di utilizzare come nucleo il kernel Linux}
}

\newglossaryentry{nodo}
{
    name=\glslink{nodo}{nodo},
    text=nodo,
    sort=nodo,
    description={Insieme di specifiche le cui implementazioni vengono sviluppate in linguaggio di programmazione Java con ampio utilizzo nella programmazione Web}
}

\newglossaryentry{machine-learning}
{
    name=\glslink{machine-learning}{machine-learning},
    text=machine-learning,
    sort=machine-learning,
    description={Campo dell'informatica che fornisce alle macchine l'abilità di imparare, nello svolgere determinati compiti, senza essere programmate esplicitamente}
}

\newglossaryentry{master}
{
    name=\glslink{master}{master},
    text=master,
    sort=master,
    description={L'unità centrale di coordinazione dei processi che possiede la configurazione, carica i plugin e gestisce tutti gli aspetti di esecuzione dei vari \textit{projects}}
}

\newglossaryentry{middleware}
{
    name=\glslink{middleware}{middleware},
    text=middleware,
    sort=middleware,
    description={Insieme di programmi informatici che fungono da intermediari tra diverse applicazioni e componenti software}
}

\newglossaryentry{pipeline}
{
    name=\glslink{pipeline}{pipeline},
    text=pipeline,
    sort=pipeline,
    description={L'insieme di passi da seguire e compiti da svolgere che coinvolgono l'intero processo di build dal recupero del codice sorgente al rilascio}
}

\newglossaryentry{plugin}
{
    name=\glslink{plugin}{plugin},
    text=plugin,
    sort=plugin,
    description={Componente software che fornisce funzionalità aggiuntive ad estensione di un determinato prodotto software preesistente ma separatamente da esso}
}

\newglossaryentry{project}
{
    name=\glslink{project}{project},
    text=project,
    sort=project,
    description={Una configurazione definita da un utente, che descrive un lavoro che Jenkins dovrà svolgere, come ad esempio costruire una componente software, eseguire uno script, ecc... A volte riferito come \textit{job}}
}

\newglossaryentry{Red Hat}
{
    name=\glslink{Red Hat}{Red Hat},
    text=Red Hat,
    sort=red hat,
    description={Società multinazionale statunitense che si dedica allo sviluppo software e al supporto di software libero e open source in ambiente aziendale}
}

\newglossaryentry{repository}
{
    name=\glslink{repository}{repository},
    text=repository,
    sort=repository,
    description={Base di dati centralizzata nella quale risiedono, individualmente, tutti i \gls{CI} di ogni \gls{baseline} nella loro storia completa}
}

\newglossaryentry{slave}
{
    name=\glslink{slave}{slave},
    text=slave,
    sort=slave,
    description={Letteralmente, schiavo. In Jenkins rappresenta un nodo sul quale vengono eseguiti più compiti, definiti sotto forma di \textit{project}, secondo la direzione di un \textit{master}. A volte definito come \textit{Agent}}
}

\newglossaryentry{srp}
{
    name=\glslink{srp}{srp},
    text=Single Responsibility Principle,
    sort=single responsibility principle,
    description={Principio della programmazione orientata agli oggetti, il quale afferma che ogni componente deve avere una e una sola responsabilità, cioè causa di cambiamento, in modo da garantire la coesione}
}

\newglossaryentry{sshg}
{
    name=\glslink{ssh}{SSH},
    text=SSH,
    sort=ssh,
    description={Protocollo che permette di stabilire una sessione remota cifrata tramite interfaccia a riga di comando con un altro host di una rete}
}

\newglossaryentry{stakeholder}
{
    name=\glslink{stakeholder}{stakeholders},
    text=stakeholders,
    sort=stakeholder,
    description={Letteralmente, ``portatore di interesse". Insieme delle persone che a vario titolo sono coinvolte nel ciclo di vita del software avendo influenza sul prodotto o sul processo; fanno parte di questa categoria fornitori, committenti e clienti}
}

\newglossaryentry{system integration}
{
    name=\glslink{system integration}{system integration},
    text=system integration,
    sort=system integration,
    description={Processo che consiste nel collegare assieme differenti sistemi e applicazioni software, a livello fisico o funzionale in modo che agiscano come un insieme coordinato e coeso}
}

\newglossaryentry{systemd}
{
    name=\glslink{systemd}{systemd},
    text=systemd,
    sort=systemd,
    description={Suite di utilità di amministrazione progettate con lo scopo di centralizzare la gestione e la configurazione dei sistemi operativi Unix-like.}
}

\newglossaryentry{war}
{
    name=\glslink{war}{war},
    text=war,
    sort=war,
    description={Formato di archiviazione che raggruppa tutti i file facenti parte di un'applicazione web in Java}
}