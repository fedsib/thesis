
%**************************************************************
% Acronimi
%**************************************************************
\renewcommand{\acronymname}{Acronimi e abbreviazioni}

\newacronym[description={\glslink{apig}{Application Program Interface}}]
    {api}{API}{Application Program Interface}

\newacronym[description={\glslink{umlg}{Unified Modeling Language}}]
    {uml}{UML}{Unified Modeling Language}

%**************************************************************
% Glossario
%**************************************************************
%\renewcommand{\glossaryname}{Glossario}

\newglossaryentry{apig}
{
    name=\glslink{api}{API},
    text=Application Program Interface,
    sort=api,
    description={in informatica con il termine \emph{Application Programming Interface API} (ing. interfaccia di programmazione di un'applicazione) si indica ogni insieme di procedure disponibili al programmatore, di solito raggruppate a formare un set di strumenti specifici per l'espletamento di un determinato compito all'interno di un certo programma. La finalità è ottenere un'astrazione, di solito tra l'hardware e il programmatore o tra software a basso e quello ad alto livello semplificando così il lavoro di programmazione}
}

\newglossaryentry{umlg}
{
    name=\glslink{uml}{UML},
    text=UML,
    sort=uml,
    description={In ingegneria del software \emph{UML, Unified Modeling Language} (ing. linguaggio di modellazione unificato) è un linguaggio di modellazione e specifica basato sul paradigma object-oriented. L'\emph{UML} svolge un'importantissima funzione di ``lingua franca'' nella comunità della progettazione e programmazione a oggetti. Gran parte della letteratura di settore usa tale linguaggio per descrivere soluzioni analitiche e progettuali in modo sintetico e comprensibile a un vasto pubblico}
}

\newglossaryentry{agent}
{
    name=\glslink{agent}{agent},
    text=agent,
    sort=agent,
    description={Sinonimo di slave. Nodo fisico, virtuale o container, che esegue compiti definiti e diretti da un'istanza \textit{master}.}
}

\newglossaryentry{artefatto}
{
    name=\glslink{artefatto}{artefatto},
    text=artefatto,
    sort=artefatto,
    description={Un prodotto immutabile, generato durante una \textit{build} che verrà opportunamente archiviato in un \textit{repository} e reso disponibile a determinati utenti per usi specifici (es. rilascio, testing).}
}

\newglossaryentry{build}
{
    name=\glslink{build}{build},
    text=build,
    sort=build,
    description={La costruzione, osservabile e tangibile di una componente software a partire dal suo codice sorgente.\\ 
    Risultato della singola esecuzione di un \textit{project}, in \textit{Jenkins}. }
}

\newglossaryentry{container}
{
    name=\glslink{container}{container},
    text=container,
    sort=container,
    description={Processo isolato dal resto del sistema che rappresenta l'istanza in esecuzione di una data immagine, secondo opportune configurazioni aggiuntive.}
}

\newglossaryentry{continuous building}
{
    name=\glslink{continuous building}{continuous building},
    text=continuous building,
    sort=continuous building,
    description={In ingegneria del software per \emph{continuous building}, (spesso identificata come ``Build automation'') si intende il processo di automazione dell'intero ciclo di build a partire dalla compilazione del codice sorgente, al garantire che i risultati siano sempre ripetibili e conformi alle attese, all'esecuzione dei test e al rendere disponibile gli artefatti risultanti. Tale processo è una parte consistente dell'\textit{integrazione continua}. }
}

\newglossaryentry{deploy}
{
    name=\glslink{deploy}{deploy},
    text=deploy,
    sort=deploy,
    description={L'insieme delle attività che permettono ad una componente software di essere disponibile ad un determinato uso.}
}

\newglossaryentry{immagine}
{
    name=\glslink{immagine}{immagine},
    text=immagine,
    sort=immagine,
    description={Base fondante e immutabile utilizzata per la costruzione di \textit{container}.}
}
 
\newglossaryentry{Java EE}
{
    name=\glslink{Java EE}{Java EE},
    text=Java EE,
    sort=Java EE,
    description={Insieme di specifiche le cui implementazioni vengono sviluppate in linguaggio di programmazione Java con ampio utilizzo nella programmazione Web.}
}

\newglossaryentry{nodo}
{
    name=\glslink{nodo}{nodo},
    text=nodo,
    sort=nodo,
    description={Insieme di specifiche le cui implementazioni vengono sviluppate in linguaggio di programmazione Java con ampio utilizzo nella programmazione Web.}
}

\newglossaryentry{master}
{
    name=\glslink{master}{master},
    text=master,
    sort=master,
    description={L'unità centrale di coordinazione dei processi che possiede la configurazione, carica i plugin e gestisce tutti gli aspetti di esecuzione dei vari \textit{projects}.}
}

\newglossaryentry{middleware}
{
    name=\glslink{middleware}{middleware},
    text=middleware,
    sort=middleware,
    description={Insieme di programmi informatici che fungono da intermediari tra diverse applicazioni e componenti software.}
}

\newglossaryentry{plugin}
{
    name=\glslink{plugin}{plugin},
    text=plugin,
    sort=plugin,
    description={Componente software che fornisce funzionalità aggiuntive ad estensione di un determinato prodotto software preesistente ma separatamente da esso.}
}

\newglossaryentry{project}
{
    name=\glslink{project}{project},
    text=project,
    sort=project,
    description={Una configurazione definita da un utente, che descrive un lavoro che Jenkins dovrà svolgere, come ad esempio costruire una componente software, eseguire uno script, ecc... A volte riferito come \textit{job}.}
}

\newglossaryentry{slave}
{
    name=\glslink{slave}{slave},
    text=slave,
    sort=slave,
    description={Letteralmente, schiavo. In Jenkins rappresenta un nodo sul quale vengono eseguiti più compiti, definiti sotto forma di \textit{project}, secondo la direzione di un \textit{master}. A volte definito come \textit{Agent}.}
}

\newglossaryentry{system integration}
{
    name=\glslink{system integration}{system integration},
    text=system integration,
    sort=system integration,
    description={Processo che consiste nel collegare assieme differenti sistemi e applicazioni software, a livello fisico o funzionale in modo che agiscano come un insieme coordinato e coeso.}
}