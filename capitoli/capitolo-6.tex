% !TEX encoding = UTF-8
% !TEX TS-program = pdflatex
% !TEX root = ../tesi.tex

%**************************************************************
\pagestyle{IHA-fancy-style}
\chapter{Verifica e validazione}
\label{cap:verifica-validazione}
%**************************************************************

\intro{In questo capitolo vengono riportati i test funzionali eseguiti durante i processi di Verifica e Validazione dell'architettura sviluppata.}

%**************************************************************
\section{Resoconto attività Testing}
%**************************************************************

In tabella \hyperref[tab:test]{6.1} vengono riportati i test di verifica eseguiti; tali test costituiscono anche una suite di collaudo del sistema. \\

Ciascun test è descritto da:

\begin{itemize}
    \item \textbf{ID}. Un codice alfanumerico che identifica univocamente il test;
    \item \textbf{Descrizione}. Una breve descrizione del test;
    \item \textbf{Comportamento atteso}. Un breve testo che specifica i risultati attesi, in caso di riuscita del test.
    \item \textbf{Stato}. Lo stato d'esecuzione del test, esso può essere:
    \begin{itemize}
        \item \textbf{Riuscito}. Se il test ha avuto esito positivo;
        \item \textbf{Fallito}. Se il test ha avuto esito negativo.
    \end{itemize}
\end{itemize}

\taburowcolors[2] 2{tableLineOne .. tableLineTwo}
\tabulinesep = ^3mm_2mm
\begin{longtabu} to \textwidth {cXXc}
    \caption[Test Funzionali]{Test Funzionali}
    \label{tab:test}
    \endlastfoot
    \rowfont{\bfseries\sffamily\leavevmode\color{white}}
    \rowcolor{tableHeaderBlue}
    %\hline
    \textbf{ID} & \textbf{Descrizione} & \textbf{Comportamento atteso} & \textbf{Stato}\\
    %\hline
    TF001 & Visualizzazione della pagina principale di Jenkins & La pagina principale di Jenkins viene visualizzata & Riuscito\\ %\hline
    TF002 & Autenticazione con l'utente admin di default & L'utente admin riesce ad autenticarsi correttamente con le sue credenziali & Riuscito\\ %\hline
    TF003 & Caricamento dei \gls{project} & Se il container viene rimosso e ne viene creato uno nuovo, con la stessa configurazione i project rimangono & Riuscito\\ %\hline
    TF004 & Caricamento dei \gls{plugin} & Se il container viene rimosso e ne viene creato uno nuovo, con la stessa configurazione i plugin rimangono & Riuscito\\ %\hline
    \newpage
    \rowfont{\bfseries\sffamily\leavevmode\color{white}}
    \rowcolor{tableHeaderBlue}
    %\hline
    \textbf{ID} & \textbf{Descrizione} & \textbf{Comportamento atteso} & \textbf{Stato}\\
    TF005 & \gls{build} di applicazioni web Maven\cite[Chapter~10]{book:maven-reference} & L'applicazione di esempio genera un file ``.\gls{war}'' & Riuscito\\ %\hline
    TF006 & \gls{build} di immagini Docker & L'immagine Docker viene costruita correttamente ed è disponibile nell'\gls{host} & Riuscito\\ %\hline
    TF007 & Esecuzione di \gls{build} distribuite\cite[Chapter~11]{book:jenkins-definitive}\cite{site:jenkins-distributed-builds} con più \gls{host} & Le \gls{build} vengono eseguite correttamente su entrambi i nodi \gls{host} & Riuscito\\ %\hline
    TF008 & Esecuzione di \gls{build} su \gls{slave} \gls{sshg} & Le \gls{build} vengono eseguite correttamente su uno \gls{slave} SSH \gls{container} & Riuscito\\ %\hline
    TF009 & Esecuzione di \gls{build} su \gls{slave} \gls{jnlpg} & Le \gls{build} vengono eseguite correttamente su uno \gls{slave} JNLP \gls{container} & Riuscito\\ %\hline
    TF010 & Recupero degli artefatti dagli \gls{slave} & Gli artefatti vengono copiati correttamente nel \textit{working space} del nodo \gls{master} & Riuscito\\ %\hline
    TF011 & Recupero del codice sorgente da un \gls{repository} \gls{cvs} remoto & La \gls{build} riesce a recuperare il codice sorgente dell'applicazione da costruire da un repository CVS remoto & Riuscito\\ %\hline
    TF012 & Aggiornamento dell'istanza \gls{master} su \gls{container} & L'aggiornamento dell'istanza master avviene correttamente & Riuscito\\ %\hline
    TF013 & Backup dei dati di Jenkins dal volume del \gls{container} \gls{master}& Viene generato un archivio contenente i dati della home directory di Jenkins & Riuscito\\ %\hline
    TF014 & Migrazione dei dati di Jenkins su un altro \gls{container} & La home directory di Jenkins viene trasferita su un nuovo container correttamente & Riuscito\\ %\hline
\end{longtabu}

%**************************************************************
\section{Conclusioni}
%**************************************************************

Tutti i test preventivati hanno riportato esito positivo anche se si evidenziano difficoltà nei test \textit{TF009} e \textit{TF010}. Tali difficoltà sono dovute principalmente a problemi col proxy aziendale.