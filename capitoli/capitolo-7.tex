% !TEX encoding = UTF-8
% !TEX TS-program = pdflatex
% !TEX root = ../tesi.tex

%**************************************************************
\pagestyle{IHA-fancy-style}
\chapter{Conclusioni}
\label{cap:conclusioni}
%**************************************************************

\intro{In questo capitolo vengono tratte le considerazioni finali circa le conoscenze acquisite e gli sviluppi futuri del progetto, riportando inoltre una valutazione personale del periodo di stage in rapporto al percorso di studi. Vengono inoltre specificati gli obiettivi raggiunti e i rischi presentatisi.}

%**************************************************************
\section{Raggiungimento degli obiettivi}
\label{sec:obiettivi-raggiunti}
%**************************************************************

Gli obiettivi esposti in \hyperref[sec:ob]{sezione 2.3} e riassunti in \hyperref[tab:obiettivi-iniziali]{tabella 2.1} non anno subito grosse variazioni rispetto quanto inizialmente preventivato. L'obiettivo \textit{O05} relativo allo studio delle modalità di comunicazione tra \gls{master} e \gls{slave}, all'inizio del progetto non era presente e con la sua introduzione si è dovuto ridimensionare l'obiettivo facoltativo \textit{F01} sullo sviluppo d'un \gls{plugin} base. Tale obiettivo facoltativo è stato parzialmente raggiunto anche se non soddisfa pienamente le aspettative aziendali. Non è stato raggiunto l'obiettivo desiderabile \textit{D02} relativo alla stesura di una relazione sui confronti prestazionali tra la piattaforma containerizzata e non, in quanto difficilmente analizzabile in sole otto settimane di stage. Le prestazioni del sistema dipendono da molti parametri (quali ad esempio i limiti imposti ai \gls{container} e le risorse a disposizione dell'\gls{host}). 

Sono stati raggiunti tutti gli obiettivi obbligatori e la quasi totalità di quelli desiderabili. In \hyperref[tab:obiettivi-finali]{tabella 7.1} vengono riassunti gli obiettivi, riportando per ciascuno lo stato di avanzamento, il quale può essere:

\begin{itemize}
    \item \textbf{Raggiunto}. Se l'obiettivo soddisfa pienamente le aspettative aziendali;
    \item \textbf{Parzialmente Raggiunto}. Se l'obiettivo è stato raggiunto ma non è pienamente conforme alle attese;
    \item \textbf{Non Raggiunto}. Se l'obiettivo non ha prodotto i risultati sperati.
\end{itemize}

\newpage
\taburowcolors[2] 2{tableLineOne .. tableLineTwo}
\tabulinesep = ^3mm_2mm
\begin{longtabu} to \textwidth {ccXc}
    \caption[Obiettivi Raggiunti]{Obiettivi Raggiunti}
    \label{tab:obiettivi-finali}
    \endlastfoot
    \rowfont{\bfseries\sffamily\leavevmode\color{white}}
    \rowcolor{darkorange}
    %\hline
    \textbf{ID} & \textbf{Priorità} & \textbf{Descrizione} & Stato \\
    %\hline
    O01 & Obbligatorio & Progettazione dell’architettura di un sistema di \gls{continuous building} & Raggiunto \\ %\hline
    O02 & Obbligatorio & Implementazione dell’architettura progettata su sistemi IKS & Raggiunto \\ %\hline
    O03 & Obbligatorio & Stesura manuale di installazione & Raggiunto \\ %\hline
    O04 & Obbligatorio & Utilizzo della piattaforma implementata per la produzione di \gls{container} Docker & Raggiunto \\ %\hline
    O05 & Obbligatorio & Studio di fattibilità sulle modalità di comunicazione master e slave & Raggiunto \\ %\hline
    D01 & Desiderabile & Containerizzazione della piattaforma implementata & Raggiunto\\ %\hline
    D02 & Desiderabile & Relazione sui confronti prestazionali tra la piattaforma e la sua controparte containerizzata & Non Raggiunto\\ %\hline
    DO3 & Desiderabile & Individuazione metriche per il monitoraggio della piattaforma & Raggiunto \\ %\hline
    DO4 & Desiderabile &  Individuazione procedure per il monitoraggio, backup, ripristino e upgrade dell’infrastruttura & Raggiunto\\ %\hline
    DO5 & Desiderabile & Breve relazione che documenti metriche e procedure dei punti DO3-4 & Raggiunto\\ %\hline
    F01 & Facoltativo & Implementazione di un plugin base per Jenkins & Parzialmente Raggiunto\\ %\hline
\end{longtabu}

Infine, in \hyperref[tab:resume]{tabella 7.2} viene sintetizzata la percentuale di raggiungimento degli obiettivi in base alla priorità associata, sul totale.

\taburowcolors[2] 2{tableLineOne .. tableLineTwo}
\tabulinesep = ^3mm_2mm
\begin{longtabu} to \textwidth {cc}
    \caption[Obiettivi Raggiunti]{Obiettivi Raggiunti}
    \label{tab:resume}
    \endlastfoot
    \rowfont{\bfseries\sffamily\leavevmode\color{white}}
    \rowcolor{darkorange}
    %\hline
    \textbf{Priorità} & \textbf{Percentuale di raggiungimento}\\
    %\hline
    Obbligatori & 100 \% \\ 
    Desiderabili & 80 \% \\ 
    Facoltativi & \textasciitilde100 \% \\ 
\end{longtabu}

%**************************************************************
\section{Resoconto dell'analisi dei rischi}
%**************************************************************

%**************************************************************
\section{Possibili Sviluppi futuri}
%**************************************************************

In questa sezione vengono evidenziati alcuni possibili sviluppi futuri del sistema implementato.\\

\subsection{Monitoraggio}
Come visto in \hyperref[sec:monitoring]{sezione 5.1}, cadvisor non permette nativamente la persistenza dei dati raccolti; inoltre sono presenti altri file di log (per esempio i log delle \gls{build} e i log degli \gls{slave} container) che sono sparsi e di difficile lettura. 

Sarebbe interessante valutare un'integrazione con l'infrastruttura ELK (elasticsearch, Logstash e Kibana) già presente in azienda, soprattutto considerato che cadvisor è integrabile con questo stack tecnologico.

\begin{figure}[H]
    \capstart
    \centering
    \includegraphics[width=7cm, keepaspectratio]{../immagini/elk.png}
    \caption{Loghi dello stack ELK}
\end{figure}

\subsection{Repository prodotti}

Scopo dello stage era l'implementazione di tutto il sistema di \gls{continuous building} ma non è mai stato fatto un vero uso degli artefatti prodotti. Un possibile sviluppo sarebbe integrare la soluzione da me sviluppata con un \gls{repository} prodotti come Nexus Repository Manager\footnote{\url{https://www.sonatype.com/nexus-repository-sonatype}}, già utilizzato con profitto dall'azienda.\\

Anche per le immagini Docker sarebbe utile approfondire l'integrazione del sistema con un \gls{registry} pubblico o privato, cioè con un repository appositamente pensato per la memorizzazione, la gestione e la distribuzione di immagini Docker.


%**************************************************************
\section{Valutazione personale}
%**************************************************************

%**************************************************************
\section{Conoscenze acquisite}
%**************************************************************