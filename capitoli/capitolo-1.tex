% !TEX encoding = UTF-8
% !TEX TS-program = pdflatex
% !TEX root = ../tesi.tex

%**************************************************************
\chapter{Introduzione}
\label{cap:introduzione}
%**************************************************************

In questo capitolo introduttivo viene fornita una panoramica dell’azienda \myCompany, proponente dello stage da me svolto e l’idea alla base del progetto.
Vengono inoltre fissate alcune convenzioni tipografiche per agevolare la lettura del documento e viene presentata la struttura dello stesso, secondo una suddivisione per capitoli.

%\noindent Esempio di utilizzo di un termine nel glossario \\
%\gls{api}. \\

%\noindent Esempio di citazione in linea \\
%\cite{site:agile-manifesto}. \\

%\noindent Esempio di citazione nel pie' di pagina \\
%citazione\footcite{womak:lean-thinking} \\

%**************************************************************
\section{L'azienda}

IKS s.r.l. (Information Knowledge Supply) nasce nel 1999 a Padova, come azienda di 
consulenza e \textit{\gls{system integration}} con servizi rivolti ad una variegata tipologia di clienti quali banche e centri servizi, industria e pubblica amministrazione.\\

L'azienda punta da sempre su temi strategici in capo informatico, quali la sicurezza informatica, lo sviluppo di soluzioni software \textit{\gls{Java EE}}, l'automazione e l'innovazione. \\

Questo approccio portò IKS negli anni 2000 ad essere uno dei leader nel settore, nel nord-est italiano, con sedi a Padova, Milano e Trento e nel 2016 a formare, assieme ad altre tre aziende (Insirio, System Evolution e Kirey) il gruppo Kirey, con l'obiettivo di:

\begin{shadequote}
    Diventare il principale progetto italiano di integrazione e collaborazione fra imprese del mondo IT e assumere il ruolo di protagonista di riferimento di tutte le aziende che vogliono affrontare percorsi di innovazione digitale, riunendo eccellenze di settore, competenze verticali e partnership tecnologiche di alto livello.\par\emph{Sito internet Kirey Group}
\end{shadequote}


\vspace{2.5em}
\begin{figure}[H]
    \capstart
    \centering
    \includegraphics[height=2cm, keepaspectratio]{../immagini/IKS.png}
    \caption{Logo di IKS s.r.l.}
\end{figure}

L'azienda è strutturata in \textit{business units (BU)}, cioè divisioni specializzate in particolari compiti. Ad esempio: la BU \textit{\gls{middleware}} si occupa della gestione della configurazione di applicativi dei clienti, andando a fornire opportune impostazioni dell'infrastruttura secondo le richieste pervenute.


%**************************************************************
\section{L'idea}
%TODO
Introduzione all'idea dello stage.

%**************************************************************
\section{Organizzazione del documento}

Dopo questo primo capitolo introduttivo, il resto del documento si articola come segue:

\begin{description}
    
    \item[{\hyperref[cap:il-progetto]{Il secondo capitolo}}] descrive in dettaglio il progetto di stage, riportando le motivazioni che hanno spinto l'azienda a proporlo e le mie personali nella scelta, gli obiettivi da raggiungere e la relativa pianificazione.
    
    \item[{\hyperref[cap:tecnologie-adottate]{Il terzo capitolo}}] approfondisce le tecnologie adottate nella realizzazione del progetto, fornendo per ciascuna di esse una panoramica generale e contestualizzandone il ruolo nel sistema implementato.
    
    \item[{\hyperref[cap:realizzazione-del-sistema]{Il quarto capitolo}}] illustra le varie fasi di realizzazione del sistema, dalla progettazione all'effettiva implementazione.
    
    \item[{\hyperref[cap:vantaggi-e-punti-di-attenzione]{Il quinto capitolo}}]riporta una panoramica approfondita sui vantaggi e i punti di attenzione dell'architettura individuata.
    
    \item[{\hyperref[cap:esercibilità]{Il sesto capitolo}}]Illustra le procedure individuate in merito all'esercibilità del sistema, cioè agli aspetti di monitoraggio, upgrade, backup e ripristino della piattaforma.
        
    \item[{\hyperref[cap:verifica-validazione]{Il settimo capitolo}}] riporta le strategie di verifica e validazione adottate e il risultato dei test condotti.
    
    \item[{\hyperref[cap:conclusioni]{Nell'ottavo capitolo}}] infine, si riportano alcune considerazioni circa l'attività di stage e gli obiettivi raggiunti.
\end{description}

Riguardo la stesura del testo, relativamente al documento sono state adottate le seguenti convenzioni tipografiche:
\begin{itemize}
	\item gli acronimi, le abbreviazioni e i termini ambigui o di uso non comune menzionati vengono definiti nel glossario, situato alla fine del presente documento. \\
    Alcune definizioni dei termini del glossario, possono essere tratte da fonti online, tra cui: \\ 
    \url{https://it.wikipedia.org/};
	%\item Alla prima occorrenza dei termini riportati nel glossario viene utilizzata la seguente nomenclatura: \emph{parola}\glsfirstoccur;
	\item i termini in lingua straniera o facenti parti del gergo tecnico sono evidenziati con il carattere \emph{corsivo}.
\end{itemize}