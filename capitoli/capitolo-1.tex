% !TEX encoding = UTF-8
% !TEX TS-program = pdflatex
% !TEX root = ../tesi.tex

%**************************************************************
\pagestyle{IHA-fancy-style}
\chapter{Introduzione}
\label{cap:introduzione}
%**************************************************************

In questo capitolo introduttivo viene fornita una panoramica dell’azienda \myCompany, proponente dello stage da me svolto e l’idea alla base del progetto.
Vengono inoltre fissate alcune convenzioni tipografiche per agevolare la lettura del documento e viene presentata la struttura dello stesso, secondo una suddivisione per capitoli.

%\noindent Esempio di utilizzo di un termine nel glossario \\
%\gls{api}. \\

%\noindent Esempio di citazione in linea \\
%\cite{site:agile-manifesto}. \\

%\noindent Esempio di citazione nel pie' di pagina \\
%citazione\footcite{womak:lean-thinking} \\

%**************************************************************
\section{L'azienda}
%**************************************************************
IKS \footnote{\url{https://www.iks.it/}} s.r.l. (Information Knowledge Supply) nasce nel 1999 a Padova, come azienda di consulenza e \textit{\gls{system integration}} con servizi rivolti ad una variegata tipologia di clienti quali banche e centri servizi, industria e pubblica amministrazione.\\

L'azienda punta da sempre su temi strategici in capo informatico, quali la sicurezza informatica, lo sviluppo di soluzioni software \textit{\gls{Java EE}}, l'automazione e l'innovazione. \\

Questo approccio portò IKS negli anni 2000 ad essere uno dei leader nel settore, nel nord-est italiano, con sedi a Padova, Milano e Trento e nel 2016 a formare, assieme ad altre tre aziende (Insirio, System Evolution e Kirey) il gruppo Kirey\footnote{\url{http://kireygroup.com/}}, con l'obiettivo di:

\begin{shadequote}
    Diventare il principale progetto italiano di integrazione e collaborazione fra imprese del mondo IT e assumere il ruolo di protagonista di riferimento di tutte le aziende che vogliono affrontare percorsi di innovazione digitale, riunendo eccellenze di settore, competenze verticali e partnership tecnologiche di alto livello.\par\emph{Sito internet Kirey Group}
\end{shadequote}


\vspace{2.5em}
\begin{figure}[H]
    \capstart
    \centering
    \includegraphics[height=2cm, keepaspectratio]{../immagini/IKS.png}
    \caption{Logo di IKS s.r.l.}
\end{figure}

L'azienda è strutturata in \textit{business units (BU)}, cioè divisioni specializzate in particolari compiti. Ad esempio: la BU \textit{\gls{middleware}} si occupa della gestione della configurazione di applicativi dei clienti, andando a fornire opportune impostazioni dell'infrastruttura secondo le richieste pervenute.

%**************************************************************
\section{L'idea}
\label{sec:idea}
%**************************************************************
Il mercato dello sviluppo software sta cambiando; mentre in passato si assisteva alla nascita di piccole medie imprese, focalizzate sullo sviluppo di uno o due prodotti, con tempi di rilascio lunghi tra una versione e l'altra, oggi la competizione è serrata. \\ Negli ultimi anni si assiste alla nascita di nuovi paradigmi: il \gls{machine-learning}, con macchine in grado di eseguire autonomamente alcune operazioni; l'\gls{iot}, con dispositivi perennemente connessi alla rete, in grado di misurare l'ambiente che li circonda e comunicare tra loro; e infine un'automazione sempre più spinta in tutti gli ambiti delle nostre vite quotidiane.\\
Da applicazioni monolitiche (un unico prodotto software dalle molte funzionalità, strettamente dipendenti e correlate tra loro), si sta passando allo sviluppo di soluzioni flessibili a microservizi, dove vari moduli con funzionalità univoche e distinte vengono composti con altri moduli, al fine di ottenere un'architettura scalabile, snella e robusta.\\

\begin{figure}[H]
    \capstart
    \captionsetup{justification=centering}
    \centering
    \includegraphics[width=9cm, keepaspectratio]{../immagini/Microservice_Architecture.png}
    \caption{Esempio d'applicazione a microservizi, immagine tratta da \\ \url{http://microservices.io/patterns/microservices.html}}
\end{figure}

In quest'ottica, va ripensato l'intero ciclo di sviluppo del software; un'azienda, anche di grandi dimensioni, dovrà mettere in campo dei processi evolutivi per non rischiare di perdere posizioni di mercato. 

IKS fornisce ai suoi clienti degli appositi servizi di consulenza, accompagnandoli nella transizione verso una metodologia di sviluppo \textit{Devops}; tale metodologia di sviluppo è un'evoluzione del modello \gls{agile}, incentrato su un'ampia automazione delle operazioni ripetitive (quali ad esempio il testing e il rilascio) e sul principio cardine di collaborazione stretta tra sviluppo, ambito \textit{operations} (cioè gestione dell'infrastruttura, mantenimento dell'ambiente di lavoro, configurazione, ecc...) e garanzia della qualità, in passato visti come mondi separati e ben distinti. \\

Scopo dello stage è lo studio e la realizzazione di un sistema di \gls{continuous building}, che in ottica \textit{Devops}, permetta un'automazione dei processi di \gls{build}, \textit{test} e \textit{rilascio} del software con un'attenzione particolare agli aspetti di portabilità del sistema, anche in previsione di un funzionamento dello stesso in ambienti \gls{cloud}.

\begin{figure}[H]
    \capstart
    \centering
    \includegraphics[height=5cm, keepaspectratio]{../immagini/Devops.png}
    \caption{Visione d'insieme della metodologia \textit{Devops}, immagine tratta da \\ \url{http://it.wikipedia.org}}
\end{figure}

%**************************************************************
\section{Organizzazione del documento}
%**************************************************************

Dopo questo primo capitolo introduttivo, il resto del documento si articola come segue:

\begin{description}
    
    \item[{\hyperref[cap:lo-stage]{Il secondo capitolo}}] descrive in dettaglio il progetto di stage, riportando le motivazioni che hanno spinto l'azienda a proporlo e le mie personali nella scelta, gli obiettivi da raggiungere e la relativa pianificazione.
    
    \item[{\hyperref[cap:tecnologie-adottate]{Il terzo capitolo}}] approfondisce le tecnologie adottate nella realizzazione del progetto, fornendo per ciascuna di esse una panoramica generale e contestualizzandone il ruolo nel sistema implementato.
    
    \item[{\hyperref[cap:realizzazione-del-sistema]{Il quarto capitolo}}] illustra le varie fasi di realizzazione del sistema, dalla progettazione all'effettiva implementazione.
    
    \item[{\hyperref[cap:vantaggi-e-punti-di-attenzione]{Il quinto capitolo}}]riporta una panoramica approfondita sui vantaggi e i punti di attenzione dell'architettura individuata.
    
    \item[{\hyperref[cap:esercibilità]{Il sesto capitolo}}]Illustra le procedure individuate in merito all'esercibilità del sistema, cioè agli aspetti di monitoraggio, upgrade, backup e ripristino della piattaforma.
        
    \item[{\hyperref[cap:verifica-validazione]{Il settimo capitolo}}] riporta le strategie di verifica e validazione adottate e il risultato dei test condotti.
    
    \item[{\hyperref[cap:conclusioni]{Nell'ottavo capitolo}}] infine, si riportano alcune considerazioni circa l'attività di stage e gli obiettivi raggiunti.
\end{description}

Riguardo la stesura del testo, relativamente al documento sono state adottate le seguenti convenzioni tipografiche:
\begin{itemize}
	\item gli acronimi, le abbreviazioni e i termini ambigui o di uso non comune menzionati vengono definiti nel glossario, situato alla fine del presente documento. \\
    Alcune definizioni dei termini del glossario, possono essere tratte da fonti online, tra cui: \\ 
    \url{https://it.wikipedia.org/};
	%\item Alla prima occorrenza dei termini riportati nel glossario viene utilizzata la seguente nomenclatura: \emph{parola}\glsfirstoccur;
	\item i termini in lingua straniera o facenti parti del gergo tecnico sono evidenziati con il carattere \emph{corsivo}. Fanno eccezione a questa norma, i nomi delle tecnologie.
\end{itemize}