% !TEX encoding = UTF-8
% !TEX TS-program = pdflatex
% !TEX root = ../tesi.tex

%**************************************************************
\pagestyle{IHA-fancy-style}
\chapter{Esercibilità}
\label{cap:esercibilità}
%**************************************************************

\intro{Questo capitolo ha lo scopo di approfondire gli aspetti legati all'esercibilità della piattaforma implementata. Verranno quindi esposte le metriche scelte per il monitoraggio e le procedure di upgrade, backup e ripristino individuate.} 

%**************************************************************
\section{Monitoraggio}
%**************************************************************

%**************************************************************
\section{Backup e Ripristino}
%**************************************************************

%**************************************************************
\section{Upgrade}
%**************************************************************

Jenkins ha due linee di sviluppo con diverse tempistiche di rilascio distinte. La prima prevede dei cicli di sviluppo con rilasci settimanali, orientati a fornire nuove funzionalità e bugfix rapidi in modo da permettere alla comunità di essere sempre aggiornata con le ultime novità, mentre la seconda denominata \gls{lts} è orientata alla stabilità, con cicli di sviluppo più lunghi e rilascio ogni 12 settimane ed è pensata soprattutto per l'ambito aziendale\footnote{Il ciclo di sviluppo della versione \gls{lts} è riportato in dettaglio nel sito ufficiale di Jenkin \url{https://jenkins.io/download/lts/}}. \\

Il team di sviluppo di Jenkins rilascia il prodotto in vari formati nativi per una variegata tipologia di distribuzioni \gls{Linux} e per Windows; mette inoltre a disposizione delle immagini per \gls{container} già preconfezionate ed infine eroga il prodotto anche sotto forma di applicazione web, in formato ``.\textit{\gls{war}}''. \\

In ambiente containerizzato, la procedura di aggiornamento di Jenkins utilizzando l'archivio ``.war'' e i volumi di Docker diventa immediata, in dettaglio, basterà associare la directory principale di Jenkins, (da qui in avanti denominata JENKINS\_HOME) ad un volume; successivamente basterà sostituire il
file "jenkins.war" precedente, con la nuova versione, nella directory dov'è presente il Dockerfile del
\gls{master} e ricostruire l'\gls{immagine} del \gls{container}.

In seguito, avendo a disposizione un backup, si potrà procedere al ripristino dei dati nel volume. \\
L'aggiornamento dei \gls{plugin}, avviene nello stesso modo.
 

